\documentclass[a4paper,11pt]{jsarticle}


% 数式
\usepackage{amsmath,amsfonts}
\usepackage{bm}
% 画像
\usepackage[dvipdfmx]{graphicx}
% url
\usepackage{hyperref}

\begin{document}


\title{システム管理者のお仕事紹介}
\author{豊嶋択斗(とよしまたくと)}
\date{\today}
\maketitle

\section{はじめに}
島根大学ものづくり部 Pim(以降、弊部)では、現役部員やOBの方が交流するチャットツールとしてMattermostというSlackライクなOSSを使用しています。
弊部ではこのMattermostを部内のサーバ上に構築して提供しており、その管理を担っているのが\textgt{システム管理者}です。
本記事ではキャラの濃いメンバーが集いし弊部のシステム管理者の一人である著者が、知られざるシステム管理者のお仕事内容について簡単に紹介していきます。

\section{自己紹介}
本題に入る前に、まずは簡単に自己紹介をします。
豊嶋といいます。
2回生で、所属は知能情報デザイン学科(いわゆる情報科)です。
好きな言葉は最安値、嫌いなものはセール価格を自称する通常価格な商品です。頭にローカルで動作する食料品カテゴリだけの価格.c〇mがインストールされている(と自称する)人です。
どうぞよろしく。

\section{システム管理者のお仕事}
システム管理者のお仕事は大きく分けて以下の5つです
\begin{enumerate}
  \item 物理サーバ管理
  \item サーバソフトウェア管理
    \begin{enumerate}
      \item Mattermost
      \item Redmine(チケット管理システム)
    \end{enumerate}
  \item ウェブサイト (\url{https://www.pim.gr.jp} など)
  \item 部内メールシステム(SaaS)
  \item Pimサイネージサービス (\url{https://www.pim.gr.jp/services/pim-signage})
\end{enumerate}

\section{物理サーバ管理}
後で説明するサーバソフトウェアを提供するためのサーバ本体の管理を指し、主に脆弱性対応や計画停電に備えたサーバの移設等を行っています。
今年は我が家にサーバ君が2週間程お泊りに来ました。
想像よりうるさく、就寝時に少し気になっていたのですが、いなくなった夜は部屋が静まり返って少し寂しい気持ちになりました。
ちなみに、サーバ機本体にはUbuntuを入れて運用しています。

\section{サーバソフトウェア管理}
次にサーバ機で動かしているサービスについて説明します。
該当サービスはMattermostとRedmineの2つです。
Uptime KumaとGrafana Cloudというサーバ死活監視システムを用いて、サーバが予期せず落ちていないか等監視しており、障害が発生したらシステム管理者が対応に当たっています。
Mattermostのマイナーバージョンアップデートはほとんどの場合15、16日辺りにリリースされるので、弊部が毎月10日に行っている部会で更新内容を紹介することもしています。

Redmineは弊部の運営メンバーがタスク管理、進捗共有のために活用しているOSSになります。

\section{ウェブサイト}
メインのサイト\url{https://www.pim.gr.jp}では弊部の紹介や活動予定を公開しています。私はしたことがありませんが、ウェブページの作成や編集、公開までの設定を行っています。ちなみに、メインのサイトはCloudflare Pagesを用いて公開されています。

\section{メール}
弊部はさくらインターネットが提供する「さくらのメールボックス」を利用しています。メールアドレスの作成やメールボックスの容量が一定値を超えそうなユーザに対して通知を行っています。

\section{Pimサイネージサービス}
Pimサイネージサービスは弊部が提供しているデジタルサイネージサービスのことです。島根大学生物資源学部3号館118室 数理データサイエンス教育研究センターにて、学内サークルの広告を掲載しています。システム管理者は、その問い合わせ対応から掲示物の管理までを担っています。デジタルサイネージの詳細は \url{https://www.pim.gr.jp/services/pim-signage} に記載されています。弊部では担当しておりませんが、一般の方向けのサービスもございますのでよろしければご覧ください。

\section{終わりに}
以上、簡単にはなりますがシステム管理者のお仕事紹介でした。この記事を読んで、少しでもシステム管理者について知っていただければ幸いです。

\end{document}
